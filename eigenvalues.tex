\documentclass{article}
\usepackage{amsmath}
\usepackage{amssymb}
\usepackage{physics}
\usepackage{amsthm}
\usepackage{graphicx}
\usepackage{caption}
\usepackage{subcaption}
\usepackage[czech]{babel}
\title{Výpočet vlastních čísel na oblasti obdelníka}
\date{}
\begin{document}
\maketitle
Začínám s rovnicí $\Delta u=-\lambda u$. Jelikož vím, že pracuji na oblasti obdelníka, mohu něco říct o závislosti x a y a přepsat $u(x,y)$ do pro nás výhodnějšího tvaru: $u(x,y)=X(x) Y(y)$ a po aplikaci Laplace mám rovnici:
\begin{equation}
   \frac{\partial^2 X}{\partial x^2} Y
      + X \frac{\partial^2 Y}{\partial y^2}=-\lambda X Y
 \end{equation}
Tento tvar je již pro výpočty vhodnější, přičemž ho nadále upravíme do tvaru:
\begin{equation}
\frac {\frac{\partial^2 X}{\partial x^2}+\lambda X}{X}=\frac{-\frac{\partial^2 Y}{\partial y^2}}{Y}
\end{equation}
Jelikož levá strana je závislá na x a pravá na y, vyplývá nám z toho, že se obě strany rovnají stejné konstantě tj.:
 \begin{equation}
\frac {\frac{\partial^2 X}{\partial x^2}+\lambda X}{X}=\frac{-\frac{\partial^2 Y}{\partial y^2}}{Y}=\alpha
\end{equation}
Z této rovnice již snadno odvodíme systém dvou rovnic:
\begin{subequations}
\begin{equation}
\frac{\partial^2 X}{\partial x^2}+(\lambda-\alpha)X=0
\end{equation}
\begin{equation}
\frac{\partial^2 Y}{\partial y^2}+\alpha Y=0
\end{equation}
\end{subequations}
Tyto rovnice snadno vyřešíme za pomocí Dirichletových okrajových podmínek. Pro případ obdelníku, jehož šířka je $a$ a výška $b$ dostáváme podmínky:
\begin{subequations}
\begin{equation}
X(0)=X(a)=0
\end{equation}
\begin{equation}
Y(0)=Y(b)=0
\end{equation}
\end{subequations}
Obecné řešení tedy je:
\begin{subequations}
\begin{equation}
X(x)=A_1 \sin{(\sqrt{\lambda-\alpha}x)}+B_1 \cos{(\sqrt{\lambda-\alpha}x)}
\end{equation}
\begin{equation}
Y(y)=A_2 \sin{(\sqrt{\alpha}y)}+B_2 \cos{(\sqrt{\alpha}y)}
\end{equation}
\end{subequations}
Po použití podmínek $X(0)=Y(0)=0$ hned vidíme, že $B_1=B_2=0$. Pro podmínku $Y(b)=0$ musí platit, že $\sqrt{\alpha}b=n\pi$, kde $n$ je přirozené číslo, tedy $\alpha=\frac{n^2\pi^2}{b^2}$. Stejným způsobem vyjádříme $\lambda$ za pomocí podmínky $X(a)=0$ a po dosazení $\alpha$ pro $m$ a $n$ přirozená čísla dostaneme vlastní čísla:
\begin{equation}
\lambda_{m,n}=\frac{m^2\pi^2}{a^2}+\frac{n^2\pi^2}{b^2}
\end{equation}
a vlastní vektory:
\begin{equation}
u(x,y)_{m,n}=A\sin{(\frac{m\pi}{a}x)}\sin{(\frac{n\pi}{b}y)}
\end{equation}
\end{document}